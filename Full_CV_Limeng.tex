%!TEX TS-program = xelatex
%!TEX encoding = UTF-8 Unicode
% Awesome CV LaTeX Template for CV/Resume
%
% This template has been downloaded from:
% https://github.com/posquit0/Awesome-CV
%
% Author:
% Claud D. Park <posquit0.bj@gmail.com>
% http://www.posquit0.com
%
%
% Adapted to be an Rmarkdown template by Mitchell O'Hara-Wild
% 23 November 2018
%
% Template license:
% CC BY-SA 4.0 (https://creativecommons.org/licenses/by-sa/4.0/)
%
%-------------------------------------------------------------------------------
% CONFIGURATIONS
%-------------------------------------------------------------------------------
% A4 paper size by default, use 'letterpaper' for US letter
\documentclass[11pt, a4paper]{awesome-cv}

% Configure page margins with geometry
\geometry{left=1.4cm, top=.8cm, right=1.4cm, bottom=1.8cm, footskip=.5cm}

% Specify the location of the included fonts
\fontdir[fonts/]

% Color for highlights
% Awesome Colors: awesome-emerald, awesome-skyblue, awesome-red, awesome-pink, awesome-orange
%                 awesome-nephritis, awesome-concrete, awesome-darknight

\definecolor{awesome}{HTML}{414141}

% Colors for text
% Uncomment if you would like to specify your own color
% \definecolor{darktext}{HTML}{414141}
% \definecolor{text}{HTML}{333333}
% \definecolor{graytext}{HTML}{5D5D5D}
% \definecolor{lighttext}{HTML}{999999}

% Set false if you don't want to highlight section with awesome color
\setbool{acvSectionColorHighlight}{true}

% If you would like to change the social information separator from a pipe (|) to something else
\renewcommand{\acvHeaderSocialSep}{\quad\textbar\quad}

\def\endfirstpage{\newpage}

%-------------------------------------------------------------------------------
%	PERSONAL INFORMATION
%	Comment any of the lines below if they are not required
%-------------------------------------------------------------------------------
% Available options: circle|rectangle,edge/noedge,left/right

\name{Limeng (Momo) Xie}{}

\address{Athens, GA, USA}

\mobile{+1 443-691-8965}
\email{\href{mailto:momosan@uga.edu}{\nolinkurl{momosan@uga.edu}}}
\github{tomatopepper}
\linkedin{Limeng (Momo) Xie}
\twitter{happymomosan}

% \gitlab{gitlab-id}
% \stackoverflow{SO-id}{SO-name}
% \skype{skype-id}
% \reddit{reddit-id}


\usepackage{booktabs}

% Templates for detailed entries
% Arguments: what when with where why
\usepackage{etoolbox}
\def\detaileditem#1#2#3#4#5{%
\cventry{#1}{#3}{#4}{#2}{\ifx#5\empty\else{\begin{cvitems}#5\end{cvitems}}\fi}\ifx#5\empty{\vspace{-4.0mm}}\else\fi}
\def\detailedsection#1{\begin{cventries}#1\end{cventries}}

% Templates for brief entries
% Arguments: what when with
\def\briefitem#1#2#3{\cvhonor{}{#1}{#3}{#2}}
\def\briefsection#1{\begin{cvhonors}#1\end{cvhonors}}

\providecommand{\tightlist}{%
	\setlength{\itemsep}{0pt}\setlength{\parskip}{0pt}}

%------------------------------------------------------------------------------



\begin{document}

% Print the header with above personal informations
% Give optional argument to change alignment(C: center, L: left, R: right)
\makecvheader

% Print the footer with 3 arguments(<left>, <center>, <right>)
% Leave any of these blank if they are not needed
% 2019-02-14 Chris Umphlett - add flexibility to the document name in footer, rather than have it be static Curriculum Vitae
\makecvfooter
  {July, 2020}
    {Limeng (Momo) Xie~~~·~~~Curriculum Vitae}
  {\thepage}


%-------------------------------------------------------------------------------
%	CV/RESUME CONTENT
%	Each section is imported separately, open each file in turn to modify content
%------------------------------------------------------------------------------



\section{Education}\label{education}

\detailedsection{\detaileditem{Phd student in Plant Biology, GPA 4.00}{2018 - Present}{University of Georgia (UGA)}{Athens, GA}{\item{Advisor: Dr. Alexander Bucksch, Computational Plant Science Lab}}\detaileditem{Master in Horticultural Science, GPA 3.81}{2014 - 2016}{Texas A\&M University (TAMU)}{College Station, TX}{\item{Certificate in Applied Statistics}\item{Advisors: Dr.Kevin Crosby \& Dr.John Jifon, Vegetable Breeding Lab}\item{Thesis: SNP Discovery and Mapping QTLs Associated with Root Traits and Morphological Traits in Tomato}}\detaileditem{Bachelor in Agronomy, GPA 3.65}{2009 - 2014}{China Agricultural University (CAU)}{Beijing, China}{\item{Thesis: Evaluation of Methane Production in Anaerobic Reactor with Sweet Potato Vine and Dairy Manure}}\detaileditem{Exchange Student}{2011 - 2012}{Saga University}{Kyushu, Japan}{\item{Major in Plant` Science and Japanese}}}

\section{Publications}\label{publications}

\textbf{Xie, L.},Burridge,J., Klepp, N., Miller, J., Lynch, J.P.,
Bucksch, A., Phenotypic spectrum: uncovering root architecture diversity
in common bean (\emph{Phaseolus vulgaris L.}) \emph{In prep}

\textbf{Xie, L.}, Klein, P., Crosby, K., \& Jifon, J. (2019). A
Genotyping-by-sequencing Single Nucleotide Polymorphism Map and Genetic
Analysis of Root Traits in an Interspecific Tomato Population,
\emph{Journal of the American Society for Horticultural Science} 144(6),
394-404.

\section{Presentations}\label{presentations}

2020 \textbf{Xie, L.},Burridge,J., Klepp, N., Miller, J.,Lynch, J.P.,
Bucksch, A., \textbf{the 7th International Horticulture Research
Conference}, \emph{Poster}: Quantifying diversity of root architecture
types within a genotype of common bean ( \emph{Phaseolus vulgaris} .L),
Online

2019 \textbf{Xie, L.}, Bucksch, A., \textbf{International Plant
Phenotyping Symposium}, \emph{Poster}: The shape of plants revealed: A
shape theoretic perspective on statistics of trait measurements,
Nanjing, China

2019 \textbf{Xie, L.},Burridge,J., Klepp, N., Miller, J., Chutoe, C.,
Saengwilai, P., Lynch, J.P., Bucksch, A., \textbf{Crops Conference},
\emph{Poster}:The Phenotypic Spectrum: Identifying Whole Role
Architecture Types in Genotypes of Common Bean ( \emph{Phaseolus
vulgaris} .L), Huntsville, AL

2018 \textbf{Xie, L.}, Liu, S, Bucksch., A., \textbf{Plant Center Fall
Retreat}, \emph{Poster}: Extracting Traits from 3D Models of Maize Root
System Architecture``, Unicoi State Park, GA

2017 \textbf{Xie, L.}, Klein, P., Crosby, K., \& Jifon, J.,
\textbf{American Society of Horticultural Science Annual Meeting},
\emph{Talk}: ``Mapping Novel QTLs Associated with Root Morphological
Traits in an Interspecific Tomato Population, Waikoloa, HI

2016 \textbf{Xie, L.}, Klein, P., Crosby, K., \& Jifon, J.,
\textbf{Annual Meeting of Texas Plant Protection Association},
\emph{Poster}: SNP Discovery and QTL Mapping for Root Related Traits in
an Interspecific Tomato Population, Bryan, TX

2015 \textbf{Xie, L.}, Crosby, K., \& Jifon, J.,\textbf{Annual Meeting
of Texas Plant Protection Association}, \emph{Poster}: Estimates of
Genetic Variance for Drought Tolerance Traits in Tomato, Bryan, TX

2015 \textbf{Xie, L.}, Crosby, K., \& Jifon, J., \textbf{American
Society of Horticultural Science Annual Meeting}, \emph{Poster}:
Estimates of Genetic Variance for Drought Tolerance Traits in Tomato,
New Orleans, LA

\section{Awards}\label{awards}

2020 Jaworski Travel Award, Department of Plant Biology, UGA\\
2020 Travel Scholarship for Root Short Course, University of Floria\\
2016 Outstanding Graduate Student Award, Texas Plant Protection
Association\\
2013 Third Prize Scholarship for Academic Excellence, CAU\\
2012 Japan Student Service Scholarship, Japan Student Service
Organization\\
2011 Third Prize Scholarship for Academic Excellence, CAU\\
2010 Second Prize Scholarship for Academic Excellenc, CAU\\
2010 National Scholarship for Encouragement, Ministry of Education in
China\\
2010 Merit Undergrate Student Award, CAU

\section{Resarch Experience}\label{resarch-experience}

\detailedsection{\detaileditem{Graduate Reserch Assistant}{2018 - Present}{Computational Plant Science Lab, UGA}{Athens, GA}{\item{Current project: The phenotypic spectrum: quantifying the diversity of root architecture in common bean}\item{Rotation projects: Developed an image processing pipeline to automatically extract root traits from 3D models of maize root system}\item{Rotation projects: Identified potential genetic regions with insulators in Bladderwort by analyzing a public RNA-sea database}\item{Rotation projects: Explored genetic variation of volatile content among 150 tomato accessions (wild, semi-domesticated, domesticated).}}\detaileditem{Lab Techician}{2017}{Vegetable Breeding Lab, TAMU}{College Station, TX}{\item{Conducted all daily operations including plant, fertilize, trellis, prune, IPM, supplies and asset acquisition to complete short term and long breeding  goals over 7 greenhouses and acre-size field}\item{Established greenhouse trials of grafted tomatoes to study yield performance in relation to various rootstock and scion combinations}\item{Screened pepper hybrids for thrip resistance and performed hybrid testing at multiple locations across Texas}\item{Crossed pepper, tomato, melon and squash to produce hybrid seed.}}\detaileditem{Graduate Research Assistant}{2014-2016}{Vegetable Breeding Lab, TAMU}{College Station, TX}{\item{Measure morphological root traits using the WinRhizo software}\item{Extracted, purified, and quantified tomato DNA for GBS library}\item{Constructed linkage map for mapping population using R/qtl, AsMap, and Joinmap}\item{Mapped 29 QTLs for 12 root and shoot traits using R/qtl, WinQTLCartographer, MapQTL and QTLNetworks}}\detaileditem{Undergraduate Research Assistant}{2012-2014}{Biomass Engineering Lab, CAU}{Beijing, China}{\item{Reutilized agricultural wastse, e.g. dairy manure and sweetpotato vine, into clean and renewable energy}\item{Evaluated the volatile fat acid of effluent using HPLC}\item{Observed dairy manure and sweetpotato vine at a ration of 2:8 could have  as stable methane production}}}

\section{Teaching \& Mentoring}\label{teaching-mentoring}

\detailedsection{\detaileditem{Teaching Assistant}{2019 Fall}{University of Georgia}{Athens, GA}{\item{Concepts in Biology (BIOL1103), ~70 students}\item{Presented lectures for 3 lab session during semester}\item{Lead field trip and graded assigned homework for class}\item{Achieved overall student approval rating of 4.38 on 5 point scale}}\detaileditem{Mentor}{2020 Summer}{University of Georgia}{Athens, GA}{\item{Lilly Adams, REU Project: Developing a new image processing pipeline to extract root trait of Arabidopsis and identifying candidate genes for root traits from GWAS}\item{Joslyn Mcklveen, CURO Project: Analyze the effect of temperature and humidity on plant development on a newly-developed mesocosm system}}\detaileditem{Mentor}{May 2017 - Dec 2017}{Texas A\&M University}{College Station, TX}{\item{Trained 8 undergraduates of pepper and tomato breeding}}}
\pagebreak

\section{Agricultural Extension}\label{agricultural-extension}

\detailedsection{\detaileditem{Intern}{Aug 2011}{Center for Chinese Agricultural Policy (CCAP)}{Beijing, China}{\item{Participated the Farmer Field School Promotion Project led by CCAP, Ministry of Agriculture, and Rand Corporation (US)}\item{Investigated application of pesticides and fertilizers in tomato productions in Hubei Province}\item{Surveyed farmers in three villages and analyzed the questionare data}}\detaileditem{Co-leader, Summer Field Program}{June 2011}{China Agricultural University}{Beijing, China}{\item{Investigated the overuse of additives in the daylily industry in Qidong County}\item{Surveyed with local farmers and small business owners (>200 people)}\item{Presented results to the local government and appeared on the local media}\item{Organized a workshop introducing alternative methods to preserve daylily for farmers }\item{Won the Excellent Investigators Team Award }}}

\section{Synergistic Activities}\label{synergistic-activities}

2020 Judge Lead for Junior Section of 72nd Georgia Science \&
Engineering Fair\\
2019 Executive Board Member of Chinese Genomics Online Meet-up
\textbar{} Responsible for inviting guest speakers and technical support
for live stream 2018 Volunteer for March for Science at Washington, DC\\
2015 Secretary, Texas A\&M Horticulture Graduate Student Council\\
2015 Volunteer, Texas A\&M Plant Breeding Symposium\\
2012 Spearker, Introduction of Chinese Culture with local schools in
Ichikikushikino-shi, kagoshima, Japan\\
2011 CAU Ambassador, Student representative for university-wide foreign
affair activities, CAU\\
2010 Scrum Half Player of College Rugby Team \textbar{} Won the
plateclass champion of college wide games, CAU\\
2010 Volunteer, Elder Care Center \textbar{} Helped take care of senior
people monthly, Beijing\\
2009 Tutor for a group of middle school students and several high school
students in science courses, Qimen \& Beijing

\section{Professional Affiliations}\label{professional-affiliations}

Member, Botanical Society of America\\
Member, American Association for Advancement of Science\\
Member, Phi Alpha Xi Honor Society\\
Member, American Society of Horticultural Science

\section{Skills}\label{skills}

\textbf{Bioinformatics}: NGS (GBS, RNA-Seq), Reads Alignment (Bowtie,
GSNAP), SNP Calling (Tophat, VCF Tools, SamTools), Gene Expression
Analysis (Cufflinks)\\
\textbf{Quantitative and Population Genetics}: Polygenetic Tree, QTL
mapping, Marker Assistant Selection, Clustering, Principle Component
Analysis\\
\textbf{Computer and Data Sciences}: Linux/Unix, Python (Pandas, Numpy,
OpenCV, Sci-py, Jupyter Notebooks), R (R/qtl, ggplot2, Rmarkdown,
Tidyverse), SQL (MariaDB, Dbeaver), Github\\
\textbf{Language Skills}: Chinese (Native), English (Fluent), Japanese
(Basic)

\end{document}
