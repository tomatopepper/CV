%!TEX TS-program = xelatex
%!TEX encoding = UTF-8 Unicode
% Awesome CV LaTeX Template for CV/Resume
%
% This template has been downloaded from:
% https://github.com/posquit0/Awesome-CV
%
% Author:
% Claud D. Park <posquit0.bj@gmail.com>
% http://www.posquit0.com
%
%
% Adapted to be an Rmarkdown template by Mitchell O'Hara-Wild
% 23 November 2018
%
% Template license:
% CC BY-SA 4.0 (https://creativecommons.org/licenses/by-sa/4.0/)
%
%-------------------------------------------------------------------------------
% CONFIGURATIONS
%-------------------------------------------------------------------------------
% A4 paper size by default, use 'letterpaper' for US letter
\documentclass[11pt,a4paper,]{awesome-cv}

% Configure page margins with geometry
\usepackage{geometry}
\geometry{left=1.4cm, top=.8cm, right=1.4cm, bottom=1.8cm, footskip=.5cm}


% Specify the location of the included fonts
\fontdir[fonts/]

% Color for highlights
% Awesome Colors: awesome-emerald, awesome-skyblue, awesome-red, awesome-pink, awesome-orange
%                 awesome-nephritis, awesome-concrete, awesome-darknight

\definecolor{awesome}{HTML}{414141}

% Colors for text
% Uncomment if you would like to specify your own color
% \definecolor{darktext}{HTML}{414141}
% \definecolor{text}{HTML}{333333}
% \definecolor{graytext}{HTML}{5D5D5D}
% \definecolor{lighttext}{HTML}{999999}

% Set false if you don't want to highlight section with awesome color
\setbool{acvSectionColorHighlight}{true}

% If you would like to change the social information separator from a pipe (|) to something else
\renewcommand{\acvHeaderSocialSep}{\quad\textbar\quad}

\def\endfirstpage{\newpage}

%-------------------------------------------------------------------------------
%	PERSONAL INFORMATION
%	Comment any of the lines below if they are not required
%-------------------------------------------------------------------------------
% Available options: circle|rectangle,edge/noedge,left/right

\name{Limeng (Momo) Xie}{}

\address{Athens, GA, USA}

\mobile{+1 443-691-8965}
\email{\href{mailto:momosan@uga.edu}{\nolinkurl{momosan@uga.edu}}}
\github{tomatopepper}
\linkedin{Limeng (Momo) Xie}
\twitter{happymomosan}

% \gitlab{gitlab-id}
% \stackoverflow{SO-id}{SO-name}
% \skype{skype-id}
% \reddit{reddit-id}


\usepackage{booktabs}

\providecommand{\tightlist}{%
	\setlength{\itemsep}{0pt}\setlength{\parskip}{0pt}}

%------------------------------------------------------------------------------



% Pandoc CSL macros
\newlength{\cslhangindent}
\setlength{\cslhangindent}{1.5em}
\newlength{\csllabelwidth}
\setlength{\csllabelwidth}{3em}
\newenvironment{CSLReferences}[3] % #1 hanging-ident, #2 entry spacing
 {% don't indent paragraphs
  \setlength{\parindent}{0pt}
  % turn on hanging indent if param 1 is 1
  \ifodd #1 \everypar{\setlength{\hangindent}{\cslhangindent}}\ignorespaces\fi
  % set entry spacing
  \ifnum #2 > 0
  \setlength{\parskip}{#2\baselineskip}
  \fi
 }%
 {}
\usepackage{calc}
\newcommand{\CSLBlock}[1]{#1\hfill\break}
\newcommand{\CSLLeftMargin}[1]{\parbox[t]{\csllabelwidth}{#1}}
\newcommand{\CSLRightInline}[1]{\parbox[t]{\linewidth - \csllabelwidth}{#1}}
\newcommand{\CSLIndent}[1]{\hspace{\cslhangindent}#1}

\begin{document}

% Print the header with above personal informations
% Give optional argument to change alignment(C: center, L: left, R: right)
\makecvheader

% Print the footer with 3 arguments(<left>, <center>, <right>)
% Leave any of these blank if they are not needed
% 2019-02-14 Chris Umphlett - add flexibility to the document name in footer, rather than have it be static Curriculum Vitae
\makecvfooter
  {July, 2022}
    {Limeng (Momo) Xie~~~·~~~Curriculum Vitae}
  {\thepage}


%-------------------------------------------------------------------------------
%	CV/RESUME CONTENT
%	Each section is imported separately, open each file in turn to modify content
%------------------------------------------------------------------------------



\hypertarget{education}{%
\section{Education}\label{education}}

\begin{cventries}
    \cventry{Phd student in Plant Biology, GPA 3.96}{University of Georgia (UGA)}{Athens, GA}{2018 - Present}{\begin{cvitems}
\item Advisor: Dr. Alexander Bucksch, Computational Plant Science Lab
\end{cvitems}}
    \cventry{Master in Horticultural Science, GPA 3.81}{Texas A\&M University (TAMU)}{College Station, TX}{2014 - 2016}{\begin{cvitems}
\item Certificate in Applied Statistics
\item Thesis: SNP Discovery and Mapping QTLs Associated with Root Traits and Morphological Traits in Tomato
\end{cvitems}}
    \cventry{Bachelor in Agronomy, GPA 3.65}{China Agricultural University (CAU)}{Beijing, China}{2009 - 2014}{\begin{cvitems}
\item Thesis: Evaluation of Methane Production in Anaerobic Reactor with Sweet Potato Vine and Dairy Manure
\end{cvitems}}
    \cventry{Exchange Student}{Saga University}{Kyushu, Japan}{2011 - 2012}{\begin{cvitems}
\item Major in Plant` Science and Japanese
\end{cvitems}}
\end{cventries}

\hypertarget{publications}{%
\section{Publications}\label{publications}}

\textbf{Xie, L.},Burridge,J., Klepp, N., Miller, J., Lynch, J.P.,
Bucksch, A., Phenotypic spectrum: uncovering root architecture diversity
in common bean (\emph{Phaseolus vulgaris L.}) \emph{In prep}

Delory, B.M., Hernandez-Soriano, M.C., Wacker, T.S., Dimitrova, A.,
Ding, Y., Greeley, L.A., Ng, J.L.P., Mesa-Marín, J., \textbf{Xie, L.},
Zheng, C. and York, L.M., 2022. A snapshot of the root phenotyping
landscape in 2021. bioRxiv.

\textbf{Xie, L.}, Klein, P., Crosby, K., \& Jifon, J. (2019). A
Genotyping-by-sequencing Single Nucleotide Polymorphism--based Map and
Genetic Analysis of Root Traits in an Interspecific Tomato Population,
\emph{Journal of the American Society for Horticultural Science} 144(6),
394-404.

\hypertarget{presentations}{%
\section{Presentations}\label{presentations}}

2022 \textbf{Xie, L.,} Burridge,J., Lynch, J.P., Bucksch, A.,
\textbf{North America Plant Phenoytping Network}, \emph{Oral}:
``Phenotypic Spectrum: Uncovering Root Architecture Diversity in Common
Bean ( \emph{Phaseolus vulgaris} .L)'', Athens, GA

2021 \textbf{Xie, L.,} Burridge,J., Lynch, J.P., Bucksch, A.,
\textbf{ASA, CSSA, SSSA International Annual Meeting}, \emph{Poster}:
``Phenotypic Spectrum: Uncovering Root Architecture Diversity in Common
Bean ( \emph{Phaseolus vulgaris} .L)'', Salt Lake City, UT

2021 Kim, C., \textbf{Xie, L.,} Bucksch, A., Seymore, L., Van Iersel,
M., \textbf{American Society of Horticultural Science Annual Meeting},
\emph{Poster}: ``Quantification of Canopy Size Using Automated
Chlorophyll Florescence Image Analysis'', Denver, CO

2019 \textbf{Xie, L.}, Bucksch, A., \textbf{International Plant
Phenotyping Symposium}, \emph{Poster}: ``The shape of plants revealed: A
shape theoretic perspective on statistics of trait measurements'',
Nanjing, China

2019 \textbf{Xie, L.},Burridge,J., Klepp, N., Miller, J., Chutoe, C.,
Saengwilai, P., Lynch, J.P., Bucksch, A., \textbf{Crops Conference},
\emph{Poster}: ``The Phenotypic Spectrum: Identifying Whole Role
Architecture Types in Genotypes of Common Bean ( \emph{Phaseolus
vulgaris} .L)'', Huntsville, AL

2018 \textbf{Xie, L.}, Liu, S, Bucksch., A., \textbf{Plant Center Fall
Retreat}, \emph{Poster}: ``Extracting Traits from 3D Models of Maize
Root System Architecture'', Unicoi State Park, GA

2017 \textbf{Xie, L.}, Klein, P., Crosby, K., \& Jifon, J.,
\textbf{American Society of Horticultural Science Annual Meeting},
\emph{Talk}: ``Mapping Novel QTLs Associated with Root Morphological
Traits in an Interspecific Tomato Population'', Waikoloa, HI

2015 \textbf{Xie, L.}, Crosby, K., \& Jifon, J., \textbf{American
Society of Horticultural Science Annual Meeting}, \emph{Poster}:
``Estimates of Genetic Variance for Drought Tolerance Traits in
Tomato'', New Orleans, LA

\hypertarget{awards}{%
\section{Awards}\label{awards}}

2022 Georgia Education Board Fellowship, UGA\\
2022 Gerald O Mott Award, Crop Science Society of America\\
2022 Best Fast-Forward Talk at North America Plant Phenotyping Network\\
2021 Best Poster Award UGA Phenomics Symposium, UGA\\
2021 Georgia Education Board Fellowship\\
2020 Palfrey Travel Award, Department of Plant Biology, UGA\\
2020 Travel Scholarship for Root Short Course, University of Florida\\
2016 Outstanding Graduate Student Award, Texas Plant Protection
Association\\
2013 Third Prize Scholarship for Academic Excellence, CAU\\
2012 Japan Student Service Scholarship, Japan Student Service
Organization\\
2011 Third Prize Scholarship for Academic Excellence, CAU\\
2010 Second Prize Scholarship for Academic Excellence, CAU\\
2010 National Scholarship for Encouragement, Ministry of Education in
China\\
2010 Merit Undergraduate Student Award, CAU

\hypertarget{resarch-experience}{%
\section{Resarch Experience}\label{resarch-experience}}

\begin{cventries}
    \cventry{North America Breeding Intern}{Bayer}{Chesterfield, MO}{2022 May - Present}{\begin{cvitems}
\item Developed a root phenotyping platform for cotton
\item Identified root traits that improve cotton performance in drought from control enviroment to field scale
\item Presented findings to stakeholders across teams
\end{cvitems}}
    \cventry{Graduate Reserch Assistant}{Computational Plant Science Lab, UGA}{Athens, GA}{2018 - Present}{\begin{cvitems}
\item Current project: The phenotypic spectrum: quantifying the diversity of root architecture in common bean
\item Rotation projects: Developed an image processing pipeline to automatically extract root traits from 3D models of maize root system
\item Rotation projects: Identified potential genetic regions with insulators in Bladderwort by analyzing a public RNA-seq database
\item Rotation projects: Explored genetic variation of volatile content among 150 tomato accessions (wild, semi-domesticated, domesticated)
\end{cvitems}}
    \cventry{Lab Techician}{Vegetable Breeding Lab, TAMU}{College Station, TX}{2017}{\begin{cvitems}
\item Conducted all daily operations including plant, fertilize, trellis, prune, IPM, supplies and asset acquisition to complete short term and long breeding  goals over 7 greenhouses and acre-size field
\item Established greenhouse trials of grafted tomatoes to study yield performance in relation to various rootstock and scion combinations
\item Screened pepper hybrids for thrip resistance and performed hybrid testing at multiple locations across Texas
\item Crossed pepper, tomato, melon and squash to produce hybrid seed
\end{cvitems}}
    \cventry{Graduate Research Assistant}{Vegetable Breeding Lab, TAMU}{College Station, TX}{2014-2016}{\begin{cvitems}
\item Measure morphological root traits using the WinRhizo software
\item Extracted, purified, and quantified tomato DNA for GBS library
\item Constructed linkage map for mapping population using R/qtl, AsMap, and Joinmap
\item Mapped 29 QTLs for 12 root and shoot traits using R/qtl, WinQTLCartographer, MapQTL and QTLNetworks
\end{cvitems}}
    \cventry{Undergraduate Research Assistant}{Biomass Engineering Lab, CAU}{Beijing, China}{2012-2014}{\begin{cvitems}
\item Reutilized agricultural wastse, e.g. dairy manure and sweetpotato vine, into clean and renewable energy
\item Evaluated the volatile fat acid of effluent using HPLC
\item Observed dairy manure and sweetpotato vine at a ration of 2:8 could have  as stable methane production
\end{cvitems}}
\end{cventries}

\hypertarget{teaching-mentoring}{%
\section{Teaching \& Mentoring}\label{teaching-mentoring}}

\begin{cventries}
    \cventry{Teaching Assistant}{University of Georgia}{Athens, GA}{2019 Fall}{\begin{cvitems}
\item Concepts in Biology (BIOL1103), ~70 students
\item Presented lectures for 3 lab session during semester
\item Lead field trip and graded assigned homework for class
\item Achieved overall student approval rating of 4.38 on 5 point scale
\end{cvitems}}
    \cventry{Mentor}{University of Georgia}{Athens, GA}{2020-2022}{\begin{cvitems}
\item Sydney Page, CURO Project: Discovery of mirconutrients content of different root architecture types in Common Bean
\item Joslyn Mcklveen, CURO Project: Analysis the effect of temperature and humidity on plant development on a newly-developed mesocosm system
\item Lilly Adams, REU Project: Developing a new image processing pipeline to extract root trait of Arabidopsis and identifying candidate genes for root traits from GWAS
\end{cvitems}}
    \cventry{Mentor}{Texas A\&M University}{College Station, TX}{May 2017 - Dec 2017}{\begin{cvitems}
\item Trained 8 undergraduates for pepper and tomato breeding
\end{cvitems}}
\end{cventries}

\hypertarget{agricultural-extension}{%
\section{Agricultural Extension}\label{agricultural-extension}}

\begin{cventries}
    \cventry{Intern}{Center for Chinese Agricultural Policy (CCAP)}{Beijing, China}{Aug 2011}{\begin{cvitems}
\item Participated the Farmer Field School Promotion Project led by CCAP, Ministry of Agriculture, and Rand Corporation (US)
\item Investigated application of pesticides and fertilizers in tomato productions in Hubei Province
\item Surveyed farmers in three villages and analyzed the questionare data
\end{cvitems}}
    \cventry{Co-leader, Summer Field Program}{China Agricultural University}{Beijing, China}{June 2011}{\begin{cvitems}
\item Investigated the overuse of additives in the daylily industry in Qidong County
\item Surveyed with local farmers and small business owners (>200 people)
\item Presented results to the local government and appeared on the local media
\item Organized a workshop introducing alternative methods to preserve daylily for farmers 
\item Won the Excellent Investigators Team Award 
\end{cvitems}}
\end{cventries}

\hypertarget{leadership-acctivities}{%
\section{Leadership Acctivities}\label{leadership-acctivities}}

2021 Ambassador of International Society of Root Research \textbar{}
Responsible for promoting and advocating root re‐ search at the social
media platform\\
2021 Panelist Speaker of Women in STEM for Outreach Activities in
Athens‐Clarke County High School\\
2021 Judge for Senior Section of 73rd Georgia Science \& Engineering
Fair\\
2020 Judge Lead for Junior Section of 72nd Georgia Science \&
Engineering Fair\\
2019 Judge for Senior Section of 71st Georgia Science \& Engineering
Fair\\
2019 Executive Board Member of Chinese Genomics Online Meet‐up
\textbar{} Responsible for inviting guest speakers and technical support
for live stream\\
2018 Volunteer in March for Science at Washington, DC\\
2015 Secretary, Texas A\&M Horticulture Graduate Student Council\\
2015 Volunteer, Texas A\&M Plant Breeding Symposium

\hypertarget{professional-affiliations}{%
\section{Professional Affiliations}\label{professional-affiliations}}

Member, Crop Science Society of America, Agronomy Society of America,
American Society of Agronomy\\
Member, North America Plant Phenotyping Network\\
Member, American Association for Advancement of Science\\
Member, American Society of Horticultural Science

\hypertarget{skills}{%
\section{Skills}\label{skills}}

\textbf{Bioinformatics}: NGS (GBS, RNA-Seq), Reads Alignment (Bowtie,
GSNAP), SNP Calling (Tophat, VCF Tools, SamTools), Gene Expression
Analysis (Cufflinks)\\
\textbf{Quantitative and Population Genetics}: Polygenetic Tree, QTL
mapping, Marker Assistant Selection, Clustering, Principle Component
Analysis\\
\textbf{Computer and Data Sciences}: Linux/Unix, Python (Pandas, Numpy,
OpenCV, Sci-py, Jupyter Notebooks), R (R/qtl, ggplot2, Rmarkdown,
Tidyverse), SQL (MariaDB, Dbeaver), Github\\
\textbf{Language Skills}: Chinese (Native), English (Fluent), Japanese
(Basic)



\end{document}
